\documentclass{article}
\usepackage{enumerate}
\usepackage{amsmath}
\usepackage{multirow}
\usepackage{mathtools}
\usepackage[utf8]{inputenc}

\setlength{\parindent}{0.0in}
\setlength{\parskip}{0.1in}

\title{TDT4171 Artificial Intelligence Methods \\ Exercise 4}
\author{Herman Schistad -- hermansc@stud.ntnu.no}
\date{13.03.2012}

\begin{document}
\maketitle

First: I found this exercise suprisingly hard. It wasn't really the algorithm
that was especially hard, but rather my implementation and perhaps too weak
knowledge in data structures. I've reached a conclusion where I've really
should have chosen some other structures for better representing the data at
hand. But the pseudo code provided in the book was "hard to read" and with
just a small bit of exercise in translating these algorithms manually I was
just out of luck this time. 

So perhaps my biggest lesson of this task is not how to construct decision
trees, but rather to implement algoritihms from pseudo code and finding
efficient data structures.

That said, I'll try to answer the question to the best of my capability and
hopefully that will show at least something. 

\section*{What can you conclude about the result you have obtained? Which {\tt
CHOOSE\-ATTRIBUTE} is better and why?}

My findings, shows that the information gaining version of the {\tt
CHOOSE\-ATTRIBUTE} algorithms, not suprisingly, is the best. It shows great
accuracy and learns quickly (a steep happy graph). 

As I've said these results are not suprising. When we do information gaining we
try to maximize the entropy of a given subtree, in order to choose where to do
our best (and most eligible) count. This is found by looking at the different
outlooks of the node and sum of these.

When we have found our best node (which we do with {\tt IMPORTANCE} we set this
as the root node and continue downwards.

Differing from the random selection this is great because it looks at the
outlooks and finds the variables which counts the most for the model.
Imagine you are going on trip and train a model with previous trips. The {\tt
CHOOSE\-ATTRIBUTE} makes a good guess on what is most important: your bank
balance or the weather in the county in question. If this is chosen by random
the root node may be the weather or the bank balance, but e.g. had we trained
it using my preferences the bank statement would be the root node (very often).

\section*{Do you get the same result if you run {\tt CHOOSE\-ATTRIBUTE several
times}

No, you will not. Since we have implemented a {\tt random\_argmax} and the
{\tt information\_gain} function sometimes yields the same result for two nodes
(e.g. it can't decide which node is the best) we chose a node at random.
Hence we will get different trees.

\section*{What happens when you run the learner based on information gain
several times?}

We will see that the previous answer is indeed correct and that the trees are
different each time (but not too different). With a good learner we will not
see too many {\emph different} trees.
It will not help to run the learned many times in order to yield a better
result.

\section*{Dicussion of code}

Please find the code attached in {\tt code.py}.

The not so finished part of the algorithm is the actual tree printing and hence
therefore the comparison (including accuracy test).
Some comments are to be found in the code - please check that the information
gain algorithm is actually yielding 0.54 and therefore working.
Also try to change the global {\tt random} variable to True and see the result
change (and the tree grow a lot bigger).

\end{document}
