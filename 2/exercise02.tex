\documentclass{article}
\usepackage{enumerate}
\usepackage{amsmath}
\usepackage[T1]{fontenc}

\setlength{\parindent}{0.0in}
\setlength{\parskip}{0.1in}

\title{Methods in Artificial Intelligence \\
        \normalsize Exercise 2}
\author{Herman Schistad - MTDT}
\date{01.02.2012}
\begin{document}
\maketitle

\section*{Part A}
\begin{enumerate}[(i)]
\item What is the set of unobserved variable(s) for a given time-slice t

The unobserved variables in the given HMM (Hidden-markov model) are the values which we (logically) 
cannot observe - and in this example the actual weather in our model. That is in the given time-slice
t the unobserved value is the weather outside.

\item What is the set of observable variables given time-slice t

The observable variables in the given HMM are the values which we are able to observe and in which we
base our calculations (the evidence variables). In a given time-slice t, the observable value is whether
we see an umbrella or not. 

\item Present the dynamic model $P(X_t|X_{t-1})$ and the observation model $P(E_t|X_t)$ as matrices

The dynamic model:

\begin{equation}
P(X_t|X_{t-1}) = \begin{bmatrix}
  0.7 & 0.3 \\
  0.3 & 0.7
\end{bmatrix}
\end{equation}

The observation model:

\begin{equation}
P(E_t|E_{t-1}) =  \begin{bmatrix}
  0.9 & 0.0 \\
  0.0 & 0.2
\end{bmatrix}
\end{equation}

\begin{equation}
P(\neg E_t|E_{t-1}) =  \begin{bmatrix}
  0.1 & 0.0 \\
  0.0 & 0.8
\end{bmatrix}
\end{equation}

\item Which assumptions are encoded in this model? Are the assumptions reasonable for this particular domain?

Let's evaluate the Markov assumptions:

{\bf First-order Markov process:} We do not have a first-order Markov process, since we know that the wheater
is dependent on external factors such as seasons (winter/summer/autumn/spring) and enviromental changes.
Based on this we can prove this sentence false: "first-order Markov process, in which the current state depends
only on the previous state and not on any earlier states". When discussing the wheather, one could also say that
the current wheather is dependent on at least two previous days (or even more).

{\bf Stationary process:} We'll say that the model is a stationary process, since the wheather is controlled
by the laws of nature. We can prove this statement valid as we have the same probability $P(R_t|R_t-1)$ for 
all time-slices t. (See observational matrix above (2))

{\bf Sensor Markov assumption:} This is a reasonable assumption, since we generate our sensor model based on
the current evidence values. That is to say: we only base our sensor model on the value observed in the 
time-slice t.

\end{enumerate}

\section*{Part B}

\begin{itemize}

\item Given the evidence $\{ Umbrella_1 = True, Umbrella_2 = True \}$:
\begin{verbatim}
$ python forward_backward.py
---- Running forward ---
Forward-message 0: [ 0.5  0.5]
Forward-message 1: [ 0.81818182  0.18181818]
Forward-message 2: [ 0.88335704  0.11664296]
\end{verbatim}

We obtain the result:
\begin{equation}
P(X_2|e_{1:2}) = 0.88335704
\end{equation}

\item Given the evidence $\{ Umbrella_1 = True, Umbrella_2 = True, Umbrella_3 = False, Umbrella_4 = True, Umbrella_5 = True \}$:
\begin{verbatim}
$ python forward_backward.py
---- Running forward ---
Forward-message 0: [ 0.5  0.5]
Forward-message 1: [ 0.81818182  0.18181818]
Forward-message 2: [ 0.88335704  0.11664296]
Forward-message 3: [ 0.19066794  0.80933206]
Forward-message 4: [ 0.730794  0.269206]
Forward-message 5: [ 0.86733889  0.13266111]
\end{verbatim}
These are the normalized forward messages and we read that
\begin{equation}
P(X_5|e_{1:5}) = 0.86733889
\end{equation}
\end{itemize}

\section*{Part C}

\begin{itemize}

\item By implementing smoothing (the 'backward' in the FORWARD-BACKWARD algorithm) we calculate the probability
of rain on the first day given the same evidence as in (4):

\begin{verbatim}
$ python forward_backward.py
(...)
---- Results ---
0: [ 0.65334282  0.34665718]
1: [ 0.88335704  0.11664296]
2: [ 0.88335704  0.11664296]
\end{verbatim}

As one can see:
\begin{equation}
P(X_1|e_{1:2}) = \langle0.883,0.117\rangle
\end{equation}

\item The probability of rain on day 1 given the sequence of observations, given the same evidence in (5) we
can calculate the following results:

\begin{verbatim}
$ python forward_backward.py
(...)
---- Running backward ---
Backward-array for step 5: [1 1] 
Backward-array for step 4: [ 0.69  0.41] 
Backward-array for step 3: [ 0.4593  0.2437] 
Backward-array for step 2: [ 0.090639  0.150251] 
Backward-array for step 1: [ 0.06611763  0.04550767] 
Backward-array for step 0: [ 0.04438457  0.02422283]
\end{verbatim}

The answer is then given as:
\begin{equation}
P(X_1|e_{1:5}) = \langle0.867, 0.133\rangle
\end{equation}
\end{itemize}

\section*{Part D}

Please see attached file {\bf $forward\_backward.py$}.
Hopefully the code is readable enough.

\end{document}
